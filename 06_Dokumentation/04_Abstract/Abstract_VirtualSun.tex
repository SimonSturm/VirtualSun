\documentclass[a4paper]{fhnwreport}

\begin{document}
%\title{Einleitung}
Humanity is facing an evergrowing crysis. The demand for energy is rising day by day and an end is not in sight. Scientists are researching all imaignable possibilities, even as we speak. A possible solution is solar energy, however before this technology can be comercially used it needs further testing, yet this proves difficult since solar light is needed to operate a solar panel, which is difficult to reproduce in an laboratory environment. To measure a solar panel and thus optimize it, a testing unit is needed, which is difficult to design due to the aforementioned problems. A solution to this is to simulate the output of a solar. This leads to the main objective, the developement of a simulator, which emulates a real photovoltaic cell. This comes with the benefits of not needing sunlight in order to operate. Furthermore, the possibility exists to simulate different operation modes of a solar panel such as ``defect'' or ``dirty'' without the need of damaging or soiling the solar cell. In addition it is possible to to mimic a change on the intensity of the light which falls onto the solar panel. In summary a simulation is vastly superior when compared to a real photovoltaic panel in regards to testing a measurement unit for the mentioned panel
%it is of no concern to the test person under which circumstances the simulator is operated,
since no enviroment factors have to be considered and the suspectibility to faults is much smaller. 

The simulation device converts the power supply voltage to 24VDC and 3.5A. A step down converter further regulates this voltage to the desired value according to characteristic curve of a solar cell. This process is efficient, wasting close to no power due to heat generation, fast and adjustable, since a microcontroller is controlling the step down converter with a small voltage in relation to the operating mode of the device and the light intensity. The output voltage and current is measured via hall-sensor, which is then sent back to the controller to adjust the regulation voltage to the step down converter. The current measurement values can be seen on an LCD Diplay at the front of the package. Furthermore, the operation mode and the light intensity can be seen on said screen. The beforementioned modes and intensity are adjusted with three buttons below the display. 

To verifiy the functionality of VirtualSun a multitude of tests had to be done. The used methods include the measurement of the no load voltage as well as the short cut current. In addition, the output current and voltage in dependance of the load resistor was recorded and compared to the characteristic curve provided by the client.  Moreover, the influence of the light intensity on the output had to be registered as well as the two operating modes. Another important factor was the heat generation of the voltage converter, which can affect the longevity of the aforementioned converter and therefore had to be annealed in order to ensure smooth operating over a long duration. 

 
{old\_Button}
\end{document}