\documentclass[a4paper]{fhnwreport}
\usepackage[onehalfspacing]{setspace}
\begin{document}
\section{Abstract}

Humanity is facing an ever-growing crisis beacause the demand for energy is rising day by day and an end is not in sight. One possible solutions is solar energy. However, before this technology can be commercially used further testing is needed,this is difficult since solar light is needed to operate a solar panel, which is challenging to reproduce in an laboratory environment. To measure a solar panel and thus optimize it, a testing unit is needed. One possibility is to simulate the output of a solar cell, which emulates a real photovoltaic cell. This device can simulate different such as ``defect'' or ``dirty'' without the need of damaging or soiling the solar cell. A change in the intensity of the light falling onto the solar panel can also be mimicked.

The simulation device converts the line voltage to 24VDC and 4.2A. A step down converter regulates this voltage down to the desired value according to the characteristic curve of a solar cell. A microcontroller controls the step down converter with a small voltage which corresponds to the operating mode of the device and the light intensity. The output voltage and current are measured by the controller, which then adjusts the regulation voltage of the step down converter. The current output values can be seen on an LCD Display at the front of the package. The screen also displays the operation mode of the device and the light intensity. The before mentioned modes and intensity are adjusted using three buttons below the display. 

To verify the functionality of VirtualSun a multitude of tests have been done. The open circuit as well as the short cut current were measured. In addition, the output current and voltage depending on the load resistor were recorded and compared to the characteristic curve provided by the client. The influence of the light intensity on the output was also registered. The heat generation of the step down converter, which can affect the longevity of the converter, was dissipated to ensure smooth operation over a long period of time. This was done by installing a heat sink onto the voltage converter and having a fan in the casing, to dissipate the heat.

%The device could not be finished within the time frame of the project. Currently the measurement circuit of the voltage converter output is not working as intended, as well as the regulation of said controller via the micro controller. 
The functionality of the circuit parts could be proven within a laboratory environment and the measurements were conducted on the step down converter instead of the devices output, to verify its function and its suitability. The regulator is capable of producing the desired current and voltage levels of 3.09A and 22V. In addition, the ripple voltage was measured and met the requirement of a maximum of 5 percent in the range of 10 to 22 Volt output voltage.

Ideas for further development are provided.
%Although the project could not be finished according to the clients wishes, the foundation to make this device successful are provided. In the next step the measurement circuit would need to be revised. The software code which controls the DA-converter should also be reviewed since it is the most essential part of regulating the step down converter by a micro controller. If this has been checked and verified, it would be advisable to install a filter at the output of the voltage converter circuit in order to reduce the ripple voltage.

 
\end{document}