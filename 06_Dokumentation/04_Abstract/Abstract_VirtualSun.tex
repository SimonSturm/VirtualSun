\documentclass[a4paper]{fhnwreport}

\begin{document}
\section{Abstract}

Humanity is facing an ever-growing crisis. The demand for energy is rising day by day and an end is not in sight. One possible solutions is solar energy. However, before this technology can be commercially used it needs further testing. Yet this proves difficult since solar light is needed to operate a solar panel, which is challenging to reproduce in an laboratory environment. To measure a solar panel and thus optimize it, a testing unit is neveded, which is difficult to design due to the aforementioned problems. One possibility is to simulate the output of a solar cell. This leads to the main objective of this project, the development of a simulator, which emulates a real photovoltaic cell. This comes with the benefits of not needing sunlight in order to operate. Furthermore, this allows to simulate different operation modes of a solar panel such as ``defect'' or ``dirty'' without the need of damaging or soiling the solar cell. In addition it is possible to mimic a change on the intensity of the light which falls onto the solar panel. In summary a simulation is vastly superior when compared to a real photovoltaic panel in regards to testing a measurement unit%it is of no concern to the test person under which circumstances the simulator is operated,
since no environment factors have to be considered and the susceptibility to faults is much smaller. 

The simulation device converts the line voltage to 24VDC and 4.2A. A step down converter regulates this voltage down to the desired value in accordance to the characteristic curve of a solar cell. This process is efficient, wasting close to no power due to heat generation, fast and adjustable, since a micro controller is controlling the step down converter with a small voltage which corresponds to the operating mode of the device and the light intensity. The output voltage and current are measured by the controller, which then adjusts the regulation voltage of the step down converter. The current output values can be seen on an LCD Display at the front of the package. The screen also displays the operation mode of the device and the light intensity. The before mentioned modes and intensity are adjusted using three buttons below the display. 

To verify the functionality of VirtualSun a multitude of tests have been done. The used methods include the measurement of the open circuit as well as the short cut current. In addition, the output current and voltage in dependence of the load resistor were recorded and compared to the characteristic curve provided by the client. Also, the influence of the light intensity on the output was registered. Another important factor was the heat generation of the step down converter, which can affect the longevity of the converter and therefore was annealed in order to ensure smooth operating over a long operation time period. This was done by installing a heat sink onto the voltage converter and a fan in the casing, which dissipates the heat within the device.

The device could not be finished within the time frame of the project. Currently the measurement circuit of the voltage converter output is not working as intended, as well as the regulation of said controller via the micro controller. However, the functionality of said circuit parts could be proved within a laboratory environment and the before mentioned measurements could still be conducted on the step down converter instead of the device's output, to verify it's function and it's suitability for this project's purpose. It was found that the regulator is capable of producing the desired current and voltage levels of 3.09A and 22V. In addition, the ripple voltage was measured and for the greater part of the spectrum it was sufficient.

Although the project could not be finished according to the clients wishes, the foundation to make this device successful are provided. In the next step the measurement circuit would need to be revised. The software code which controls the DA-converter should also be reviewed since it is the most essential part of regulating the step down converter by a micro controller. If this has been checked and verified, it would be advisable to install a filter at the output of the voltage converter circuit in order to reduce the ripple voltage.

 
\end{document}