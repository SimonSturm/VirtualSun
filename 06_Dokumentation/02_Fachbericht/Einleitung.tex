%\documentclass[a4paper]{fhnwreport} %Legt grundlegende Formatierungen wie 
%
%\graphicspath{{./Anhang/}}%Change according to graphics folder!
%
%\begin{document}

\section{Einleitung}

Der Aufbau einer Prüfstation für Solarzellen ist aufwändig. Da im zukünftigen Projekt 4 eine solche Station entwickelt werden soll, ist es notwendig eine Solarzelle als Messobjekt zu benutzen. Das ausmessen einer solchen ist in 
der Praxis jedoch umständlich, da die Umgebungsfaktoren die Messung stark verfälschen würde.

%Es muss nicht nur ein Prüfgerät entwickelt werden, es muss weiterhin eine Solarzelle unter Laborbedingungen in Betrieb genommen werden. Dies ist aus mehreren Gründen unpraktisch und praxisuntauglich: Es muss eine Bestrahlungsquelle angefertigt werden, die die Bestrahlung der Sonne möglichst gut nachahmen kann. Auch ist das testen des Prüfgerätes erheblich schwieriger, denn es sind nun mindestens drei Fehleranfällige Schaltungsteile vorhanden, welche die Messresultate verfälschen können. 

Eine einfachere und genauere Lösung ist es, die Solarzelle mittels eines entsprechenden Gerätes zu simulieren. Dies ist das Ziel dieses Projektes: VirtualSun soll eine Solarzelle simulieren, welche einen Kurzschlussstrom von 3.09 Ampere und einer Leerlaufspannung von 22.0 Volt hat. VirtualSun kann die Bestrahlungsstärke mittels zwei Tastern im Bereich von 20\% bis 100\% verändern und zeigt sämtliche Werte auf einem LCD Display an der Frontseite des Gehäuses an. VirtualSun ist in jeder Laborumgebung einsatzfähig, da es ab Netz gespiesen werden kann und für die Ausgänge herkömmliche Buchsenstecker verwendet, an denen eine Last angeschlossen werden kann. VirtualSun ist deshalb auch kompatibel mit anderen Simulationsgeräten, da sämtliche Ausgänge potentialfrei sind. Es können also mehrere Simulationen in Serie geschaltet werden  VirtualSun ist weiterhin energiesparend; es wird ein effizienter Schaltregler verwendet. 

Dieser Bericht gliedert
 %die Überlegungen bei der Entwicklung dieses Gerätes dargelegt. Dabei wird auf 
die Bereiche: Theoretische Grundlagen, Hardware, Software und Validierung. Theoretische Grundlagen behandelt die Grundkenntnisse, die notwendig sind, um solch ein Gerät zu dimensionieren, in Hardware werden die einzelnen Bau- und Schaltungsteile, aus denen VirtualSun aufgebaut wurde beschrieben, Software erläutert im Detail, wie die Steuerung des Gerätes mittels Mikrocontroller realisiert worden ist. Validierung beschäftigt sich schlussendlich mit dem Austesten und der Funktionalität der einzelnen Baugruppen.

%\end{document}