\documentclass{fhnwreport}
\usepackage[ngerman]{babel}
\usepackage[T1]{fontenc}
\usepackage[utf8]{inputenc}
\usepackage{tikz}
\usetikzlibrary{arrows}
\usepackage{amsmath}
\usepackage{lmodern}
\usepackage[final]{pdfpages}
\usepackage{graphicx}
\usepackage{textcomp}
\usepackage{multirow}
\usepackage{todonotes}
\usepackage{pdfpages}
\bibliographystyle{IEEEtran}
\usepackage{cite}
\usepackage[nottoc,notlof,notlot,numbib]{tocbibind}
%\usepackage[nottoc, numbib]{tocbibind}



\title{%
  \textsc{Laborfähiger Photovoltaiksimulator}\\[2ex]
  \textsc{Sturm Frei Jörg}\\[2ex]
  \textsc{Fachbericht}}
\author{%
  \textsc{Projekt 3}\\
  \textsc{Team 3}}
\date{%
  \textsc{Windisch, DD.MM.YYYY}}

\begin{document}
\maketitle

\vfill
%\hspace{4em}
\textsc{%
\begin{tabbing}
tab1 \= tab2 \= tab3 \= tab4 \kill
Auftraggeber:  \>\>\>  Hans Gysin \\[2ex]
Betreuer:  \>\>\>  Anita Gertiser \\[2ex]
Experten:  \>\>\>  Hans Gysin, Peter Ganzmann, Matthias Meier \\[2ex]
Team:  \>\>\> Simon Sturm \\ 
\>\>\> Yanick Frei \\
\>\>\> Claudius J"org \\[2ex]
Studiengang: \>\>\> Elektro- und Informationstechnik
\end{tabbing}}

\clearpage
{\Large \textbf{Abstract}}

Regler werden heutzutage oft mit simplen Faustformeln dimensioniert. Für stabilere Regelungen kann die aufwändigere Phasengangmethode angewendet werden. Das Ziel dieser Arbeit ist es mit einem Programm die Phasengangmethode zu automatisieren. Diese Automatisierung wurde im Programm implementiert. Zusätzlich kann das Programm die Resultate mit verschiedenen Faustformeln vergleichen. Der Fachbericht setzt sich zusammen aus einem mathematischen und einem programmtechnischen Teil. Der mathematische Abschnitt behandelt die Streckenanalyse, die Phasengangmethode und ihr Ablauf, die Übertragungsfunktion der Regler, die Schrittantwort der Regelung und die Optimierungsmöglichkeiten. Er geht auch auf die Dimensionierung mit Faustformeln ein. Der programmtechnische Bericht erläutert die Software, die Benutzerschnittstelle und die Anwendung von Model-View-Controller.


\newpage
\includepdf[pages=1-2]{aufgabenstellung.pdf}

\newpage
\tableofcontents
\newpage

\include{einleitung}

\include{theoretischegrundlagen}

\include{javasoftware}

\include{schlusswort}

\section{Ehrlichkeitserklärung}
Hiermit erklärt der Unterzeichnende (Projektleiter), dass die vorliegende Arbeit selbstständig, ohne Hilfe Dritter und nur unter Benutzung der angegebenen Quellen verfasst worden ist.\newline
\\

\begin{tabbing}
tab1 \= tab2 \= tab3 \= tab4 \= tab5 \=tab6 \= tab7 \= tab8 \kill
Simon Sturm (Projektleiter): \>\>\>\>\>\>\>\_\_\_\_\_\_\_\_\_\_\_\_\\
\\
\\
Datum, Ort: \>\>\>\>\>\>\>\_\_\_\_\_\_\_\_\_\_\_\_, \_\_\_\_\_\_\_\_\_\_\_\_\_\_\_\_\_\_\\
\end{tabbing}



\newpage
\bibliography{literaturverzeichnis}
\listoffigures
\listoftables


\newpage
\section{Anhang}
\begin{itemize}
\item Herleitungen Formeln
\item Beispielberechnung Phasengangmethode mithilfe von Matlab
\item Bedienungsanleitung des Programmes 
\item Klassendiagramm
\item CD-ROM

\end{itemize}
\includepdf[pages=1-2]{Herleitungen_Anhang.pdf}
\includepdf[pages=1-3]{Matlab_bsp_berechnung_Anhang.pdf}
\includepdf[pages=1-2]{Bedienungsanleitung.pdf} 


\end{document}

