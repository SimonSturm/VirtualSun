%\documentclass[a4paper]{fhnwreport}
%
%\begin{document}
\section{Schlusswort}

VirtualSun konnte nicht erfolgreich abgeschlossen werden. Die theoretischen Grundlagen bezüglich der Ausgangskennlinie, sowie der verschmutzten und der defekten Solarzelle sind vorhanden. Des weiteren sind diese Kennlinien bereits in der Software umgesetzt worden. Der Schaltregler konnte bis auf den Kurzschlussfall in Betrieb genommen werden und auch der Code für die Anzeige, sowie für die Bedienknöpfe konnte fertiggestellt werden und in einer Laborumgebung getestet werden. 

Es war jedoch nicht möglich die Ansteuerung des DA-Wandlers, welcher die Regelung des Schaltwandlers übernimmt, so fertigzustellen, dass diese Regelung im Betrieb funktioniert. Die Software dafür müsste auf Fehler überprüft werden und die Anschlüsse des Wandlers verifiziert werden. Auch funktionierte die Messung des Schaltregler-Ausganges auf dem Messprint nicht mehr, diese konnte jedoch zuvor in einer Laborumgebung ausgetestet werden und konnte validiert werden. Es müsste nun der Messprint auf Fehler geprüft werden.
%Die folgenden Bereiche konnten jedoch nicht in einen funktionsfähigen Betriebszustand gebracht werden: Die Ansteuerung des DA-Wandlers, sowie die Messung des Regler Ausganges. 


%Diese beiden Schaltungsteile wurden in der Theorie konzipiert und dimensioniert, konnten jedoch aufgrund Zeitmangels nicht mehr in Betrieb genommen werden. 

Die Gründe für den Zeitmangel sind zum einen die lange Einarbeitungsphase in das Thema, als auch eine Fehlplanung der einzelnen Arbeitspakete. Dies führte dazu, dass Arbeiten ineffizient erledigt wurden und so viel Zeit verloren ging. Eine weitere Ursache war, dass die Zusammenarbeit im Team so nicht funktionierte wie sie sollte und die Kommunikation zwischen den Teammitgliedern nicht stimmte. Es kann jedoch gesagt werden, dass das Klima im Team gut war und so keine Probleme aufgrund persönlicher Konflikte entstanden. 

Obschon das Endprodukt nicht zu Terminende fertiggestellt werden konnte, konnte der Grossteil der benötigten Schaltungsteile aufgebaut werden. Was noch fehlt ist die Fehlersuche in den einzelnen Bereichen. Ein weiterführendes Projekt könnte mit dem bereits entwickelten in angemessener Zeit ein funktionierendes Gerät fertigstellen.

%\end{document}
