\section{Validierung}

\subsection{Hardware}

\subsubsection{Messschaltung}
Die Validierung der Messschaltung gestaltet sich durchaus einfach, da die beiden Formeln für die erwarteten Messwerte $messU$ (\ref{eq:messU}) und $messI$ (\ref{eq:messI}) bekannt sind. Problematisch an diesen beiden Formeln ist jedoch, dass mit idealen Bauteilen gerechnet wurde, welche in der Realität nicht verfügbar sind. Bei der Spannungsmessung sind dies lediglich die Toleranzen für die Widerstandswerte, sodass sich bei Widerständen mit 1\% Genauigkeit ein maximaler Fehler von 2.02\% bemerkbar machen kann. Dieser Fehler setzt sich jedoch linear fort, sodass weiterhin durch einen konstanten Faktor dividiert werden kann. Die dazugehörige Messreihe ist im Anhang unter \ref{subsec_messu} auf Seite \pageref{subsec_messu} aufgeführt.

Bei der Strommessung sind es sämtliche Widerstände, die beiden Operationsverstärker sowie der Hallsensor, welche Toleranzen aufweisen. Die beiden 1k$\Omega$ Widerstände, welche die Subtrahendensspannung von Vcc/2 erzeugen, würden dabei einen Offset erzeugen, welcher in der nachfolgenden invertierenden Verstärkerschaltung noch verstärkt wird. Die Toleranzen der anderen Widerstände sowie die Toleranz des Hallsensors wirken dabei als ein konstanter Vorfaktor. Der Offset sowie der Vorfaktor wurden im Anhang unter \ref{subsec_messi} auf Seite \pageref{subsec_messi} mittels einer Messreihe bestimmt.

In den Tests funktionierten die einzelnen Schaltungsteile. Das Zusammenspiel mit der restlichen Schaltung verursacht jedoch vor allem bei der Strommessung noch grössere Probleme, sodass dort weitere Zeit zur Fehlersuche investiert werden müsste.

\subsubsection{Regler}

Zur Validierung des Reglers wurde der Ausgangsstrom sowie die Ausgangsspannung bei maximaler Aussteuerung und variablem Ausgangs-Widerstand gemessen, sowie die Kennlinie bei einem fixen Lastwiderstand in Abhängigkeit der Bestrahlung:

\begin{figure}[h]
\includegraphics[width=1.0\textwidth]{ReglerMessung.png}%
\caption{Messung der Ausgangskennlinien graphisch dargestellt}
\label{fig::Reglermessung}
\end{figure}

Der Kurzschlussstrom konnte nicht gemessen werden, da der Regler sich zu stark erwärmte und unbrauchbar wurde.

Auch wurde die Effizienz des Reglers, sowie der Rippel gemessen.
Bei der Messung der Rippelspannung wurde gestgestellt das diese im Bereich 2.5-1.3V Feedbackspannung mehr als 5\% beträgt. Dieser Bereich entspricht ca. 0-55\% Bestrahlung. Um dies zu korrigieren gibt es mehrere Möglichkeiten: man kann einen grösseren Ausgangskondensator verwenden oder die Spulengrössen verändern. Es ist zudem nach anzumerken das in dieser Messchaltung kein zusätzliches Ausgangsfilter verwendet wurde (zu sehen in Grafik \ref{fig::fig::LTSchemata}), welches den Rippel der Ausgangsspannung auch verbessern würde.


%Zur Validierung des Reglers wurden die Ausgangskennlinie bei maximaler Bestrahlung in Abhängigkeit des Lastwiderstandes aufgenommen. Desweiteren wurde diese Kennlinie auch bei der minimalen Bestrahlung gemessen, um zu verifizieren ob der Strom proportional zur Bestrahlung abnimmt. Weiter wurde das Verhältnis Ausgang:Eingang gemessen, um die Effizienz des reglers zu bestimmen. Auch wurde der Rippel des Regler-Ausganges bestimmt.