\documentclass[a4paper]{fhnwreport} %Legt grundlegende Formatierungen wie 

\graphicspath{{./Anhang/}}%Change according to graphics folder!

\begin{document}

\section{Einleitung}

Der Aufbau einer Prüfstation für Solarzellen ist aufwändig. Es muss nicht nur ein Prüfgerät entwickelt werden, es muss weiterhin eine Solarzelle unter Laborbedingungen in Betrieb genommen werden. Dies ist aus mehreren Gründen unpraktisch und praxisuntauglich: Es muss eine Bestrahlungsquelle angefertigt werden, die die Bestrahlung der Sonne möglichst gut nachahmen kann. Auch ist das testen des Prüfgerätes erheblich schwieriger, denn es sind nun mindestens drei Fehleranfällige Schaltungsteile vorhanden, welche die Messresultate verfälschen können. 

Eine einfachere, bequemere und genauere Lösung ist es die Solarzelle mittels eines entsprechenden Gerätes zu simulieren. Dies ist das Ziel dieses Projektes: VirtualSun simuliert eine Solarzelle mit einem Kurzschlussstrom von 3.09 Ampere und einer Leerlaufspannung von 22.0 Volt. VirtualSun kann die Bestrahlungsstärke mittels zwei Tastern im Bereich von 20\% bis 100\% verändern und zeigt sämtliche Werte auf einem LCD Display an der Frontseite des Gehäuses an. VirtualSun ist in jeder Laborumgebung einsatzfähig, da es ab Netz gespiesen werden kann und für die Ausgänge herkömmliche Buchsenstecker verwendet, an denen eine Last angeschlossen werden kann. VirtualSun ist deshalb auch kompatibel mit anderen Simulationsgeräten, da sämtliche Ausgänge potentialfrei sind. Es können also mehrere Simulationen in Serie geschaltet werden  VirtualSun ist weiterhin energiesparend; es wird ein effizienter Schaltregler verwendet. 

In diesem Bericht werden die Überlegungen bei der Entwicklung dieses Gerätes dargelegt. Dabei wird auf die einzelnen Bereiche Theoretische Grundlagen, Hardware, Software und Validierung eingegangen. Theoretische Grundlagen behandelt die Grundkenntnisse die notwendig sind um solch ein Gerät zu dimensionieren, Hardware  beschreibt die einzelnen Bau- und Schaltungsteile aus denen VirtualSun aufgebaut wurde. Software beschreibt im Detail, wie die Steuerung des Gerätes mittels Mikrocontroller realisiert worden ist. Validierung beschäftigt sich schlussendlich mit dem Austesten und wie die einzelnen Teile auf ihre Funktion verifiziert wurde.
\newpage
%Für Laborversuche werden oft Schaltungsteile simuliert, da diese in der Praxis aufwändig zu bedienen sind. So zum Beispiel eine Solaranlage. Es soll ein PV-Modul Simulator entwickelt werden mit dem die Ausgangskennlinie einer solchen simuliert werden kann.
%
%Das Gerät ist in einem Laborgehäuse eingebaut und verfügt über zwei Laborbuchsen als Anschluss für eine Last. Desweiteren kann die Bestrahlungsstärke im Bereich von 20\% bis 100\% Bestrahlung eingestellt werden, die Einstellung erfolgt über zwei Knöpfe die an der Frontplatte des Gehäuses angebracht sind. Mittels LCD-Display werden die momentanen Werte der Bestrahlungsstärke angezeigt.
%
%Dieser Bericht beschreibt die Problemlösung, aufgeteilt in die Bereiche Theoretische Grundlagen, Hardware, Software sowie Validierung.

\section{Theoretische Grundlagen}

In diesem Kapitel werden die theoretischen Grundlagen der Solarzelle und der Mathematik genauer erläutert, welche zur Dimensionierung der Kennlinie einer Solarzelle notwendig ist. Weiterhin wird detaillierter auf den Aufbau sowie die Funktionsweise eines Schaltreglers eingegangen. 

\section{Hardware} 

Im folgendem Abschnitt werden die hardwarespezifischen Probleme erläutert, wie diese angegangen und gelöst wurden und weshalb diese Lösung gewählt worden ist. Dabei wird auf die einzelnen Hardwarebauteile genauer eingegangen: Schaltregler, LCD Display, Netzteil und Arduino UNO. Es werden die Gedankenvorgänge dargelegt, welche gemacht wurden damit die jeweiligen Schaltungsteile funktionieren (Spannung und Strombelastung, Kühlung, Platzierung auf dem Print, Massenplatzierungen).

\section{Software}

In diesem Teil wird genauer auf die Programmierung des Arduino sowie die Ansteuerung der einzelnen Schaltungsteile eingegangen. Es wird erläutert wie die Kennlinie als C Code realisiert wurde, wie das LCD Display angesteuert worden ist und wie der Schaltregler der Kennlinie entsprechend geregelt wird. 

\section{Validierung}

Diese Sektion behandelt das Testverfahren, nachdem das Produkt auf seine Funktionsweise überprüft wurde. Dabei wird die Ausgangskennlinie des Gerätes bei variabler Last aufgenommen, damit diese mit den Vorgaben des Auftraggebers verifiziert werden kann. Des weiteren wird die Kühlung der Bauteile auf ihre Tauglichkeit überprüft, damit die Funktion des Gerätes auch über lange Zeiten garantiert werden kann. 
\end{document}









