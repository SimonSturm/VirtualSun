\section{Validierung}

\subsection{Hardware}

\subsubsection{Messschaltung}
Die Validierung der Messschaltung gestaltet sich durchaus einfach, da die beiden Formeln für die erwarteten Messwerte $messU$ (\ref{eq:messU}) und $messI$ (\ref{eq:messI}) bekannt sind. Problematisch an diesen beiden Formeln ist jedoch, dass mit idealen Bauteilen gerechnet wurde, welche in der Realität nicht verfügbar sind. Bei der Spannungsmessung sind dies lediglich die Toleranzen für die Widerstandswerte, sodass sich bei Widerständen mit 1\% Genauigkeit ein maximaler Fehler von 2.02\% bemerkbar machen kann. Dieser Fehler setzt sich jedoch linear fort, sodass weiterhin durch einen konstanten Faktor dividiert werden kann. Die dazugehörige Messreihe ist im Anhang unter \ref{subsec_messu} auf Seite \pageref{subsec_messu} aufgeführt.

Bei der Strommessung sind es sämtliche Widerstände, die beiden Operationsverstärker sowie der Hallsensor, welche Toleranzen aufweisen. Die beiden 1k$\Omega$ Widerstände, welche die Subtrahendensspannung von Vcc/2 erzeugen, würden dabei einen Offset erzeugen, welcher in der nachfolgenden invertierenden Verstärkerschaltung noch verstärkt wird. Die Toleranzen der anderen Widerstände sowie die Toleranz des Hallsensors wirken dabei als ein konstanter Vorfaktor. Der Offset sowie der Vorfaktor wurden im Anhang unter \ref{subsec_messi} auf Seite \pageref{subsec_messi} mittels einer Messreihe bestimmt.

In den Tests funktionierten die einzelnen Schaltungsteile. Das Zusammenspiel mit der restlichen Schaltung verursacht jedoch vor allem bei der Strommessung noch grössere Probleme, sodass dort weitere Zeit zur Fehlersuche investiert werden müsste.

\subsubsection{Regler}\label{subsec:ValRegler}

Zur Validierung des Reglers wurde der Ausgangsstrom sowie die Ausgangsspannung bei maximaler Aussteuerung und variablem Ausgangswiderstand gemessen. Dabei kommt die Stromlimitierung nicht vom Schaltregler selbst, sondern von der zu testzwecken verwendeten Quelle. Ausserdem wurden Strom und Spannung im Verhältnis bei drei verschiedenen Lastwiderständen gemessen. Die genauen Messreihen finden sich im Anhang unter \ref{ValidMeas} auf Seite \pageref{ValidMeas}. Der Kurzschlussstrom konnte nicht gemessen werden, da der Regler sich zu stark erwärmte und unbrauchbar wurde.

Die vom Auftraggeber gestellte Bedingung an ein effizientes Gerät konnte erfüllt werden. Die Verluste sind deutlich geringer als bei Linearreglern mit gleicher Funktion. Abbildung \ref{fig:Wirkungsgrad} zeigt den Wirkungsgrad als Funktion der Ausgangsspannung. Dabei wurden die Messungen bei 20$\Omega$, 6$\Omega$ und 3$\Omega$ durchgeführt. Der seltsame Verlauf der Kennlinie bei 3$\Omega$ rührt daher, dass die Versorgungsspannung der Spannungsquelle zu testzwecken zusammengebrochen ist.

\begin{figure}[h]
	\centering
		\includegraphics[width=1.00\textwidth]{EffizienzMessung.png}
	\caption{Der Wirkungsgrad des Schaltreglers als Funktion der Ausgangsleistung.}
	\label{fig:Wirkungsgrad}
\end{figure}

Ebenfalls wichtig ist die Messung der Rippelspannung am Ausgang. Abbildung \ref{fig:RippelMessung} zeigt den prozentualen Rippel in Abhängigkeit der Ausgangsleistung. Wie auch beim Wirkungsgrad wurden diese Messungen für 20$\Omega$, 6$\Omega$ und 3$\Omega$ durchgeführt. \\
Der Rippel liegt im niedrigen Leistungsbereich über den vom Auftraggeber geforderten 5\%. Um dies zu korrigieren gibt es mehrere Möglichkeiten: Man kann einen grösseren Ausgangskondensator verwenden oder die Spulengrösse verändern. Es ist zudem noch anzumerken, das in dieser Regelschaltung kein zusätzliches Ausgangsfilter verwendet wurde (zu sehen in Grafik \ref{fig::LTSchemata}), welches den Rippel der Ausgangsspannung auch verbessern würde. In diesem Bereich ist auch der Wirkungsgrad klein. Die Messreihen zur Verzifizierung dieser Daten findet sich im Anhang unter (siehe Messreihe im Anhang \ref{ValidMeas}).

\begin{figure}[h]
	\centering
		\includegraphics[width=1.00\textwidth]{RippelMessung.png}
	\caption{Der prozentuale Ausgangsrippel als Funktion der Ausgangsleistung}
	\label{fig:RippelMessung}
\end{figure}


\paragraph{Bei der Hardware müssten für eine korrekte Funktion folgende Verbesserungen vorgenommen werden:}
\begin{itemize}
	\item Am Ausgang des Operationsverstärkers für die Strommessung (siehe Abbildung \ref{fig:Messschaltung_I}, dort als $messI$ bezeichnet) ist die Spannung zum Teil in einem undefinierten Zustand. Es wird vermutet, dass diese Störung von der ebenfalls auf dieser Platine erzeugten 5V-Hilfsspannung kommen. Bei Tests mit externer Spannungsspeisung konnten diese Fehler nicht nachvollzogen werden.
\end{itemize}