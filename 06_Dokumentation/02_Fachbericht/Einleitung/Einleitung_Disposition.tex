\documentclass[a4paper]{fhnwreport} %Legt grundlegende Formatierungen wie Schriftarten, Ort Seitenzahlen etc. fest.

%-----------------------------Wichtigste/Zentrale Packages------------------------ 
\usepackage[german]{babel}
%\usepackage[latin9]{inputenc}
\usepackage[babel, german=quotes]{csquotes}
\usepackage{hyperref}
\usepackage{verbatim}
\usepackage{amsmath}		%Mathe-Package
\usepackage{amsthm}			%Mathe-Package
\usepackage{graphicx} 	%Paket f�r die Darstellung von Abbildungen
\usepackage{fancyhdr}
\usepackage[latin1]{inputenc}
\usepackage[T1]{fontenc}
\usepackage{pdfpages}
\usepackage{tikz}
\usetikzlibrary{arrows}
\usepackage{lmodern}
\usepackage{listings}
\usepackage{pdfpages}%Required to include pdf, also possible with 				\includegraphics
\usepackage[toc,page]{appendix}%Required for appendix

\graphicspath{{./graphics/}}%Change according to graphics folder!

\begin{document}

\section{Einleitung}
F�r Laborversuche werden oft Schaltungsteile simuliert, da diese in der Praxis aufw�ndig zu bedienen sind. So zum Beispiel eine Solaranlage. Es soll ein PV-Modul Simulator entwickelt werden mit dem die Ausgangskennlinie einer solchen simuliert werden kann.

Das Ger�t ist in einem Laborgeh�use eingebaut und verf�gt �ber zwei Laborbuchsen als Anschluss f�r eine Last. Desweiteren kann die Bestrahlungsst�rke im Bereich von 20\% bis 100\% Bestrahlung eingestellt werden, die Einstellung erfolgt �ber zwei Kn�pfe die an der Frontplatte des Geh�uses angebracht sind. Mittels LCD-Display werden die momentanen Werte der Bestrahlungsst�rke angezeigt.

Dieser Bericht beschreibt die Probleml�sung, aufgeteilt in die Bereiche Theoretische Grundlagen, Hardware, Software sowie Validierung.

\section{Theoretische Grundlagen}

In diesem Kapitel werden die Theoretischen Kenntnisse genauer erl�utert, welche zur Dimensionierung der Kennlinie einer Solarzelle notwendig ist sowie die Funktionsweise und Dimensionierung eines Schaltreglers.

\section{Hardware} 

Im folgendem Abschnitt werden die Hardwaretechnischen Probleme erl�utert, wie diese angegangen und gel�st wurden und weshalb diese L�sung gew�hlt worden ist.

\section{Software}

In diesem Teil wird erl�utert auf wie die Erstellung der Software angegangen wurde, wie eventuelle Probleme gel�st worden sind und weshalb dies so getan wurde.

\section{Validierung}

Diese Sektion behandelt das Testverfahren, nachdem das Produkt auf seine Funktionsweise �berpr�ft wurde. 
\end{document}









