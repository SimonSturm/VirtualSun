\subsection{Bedienung}
Die Bedienungssoftware ist anwenderfreundlich aufgebaut, indem der Benutzer in zwei Schritten die Bestrahlungsstärke und den Modus wählt. Zeitgleich gibt die Software die Einstellungen des Benutzers und die berechneten Spannungs- und Stromwerte auf dem Bildschirm aus.

Das LCD-Display zeigt in der Reihenfolge Spannung, Strom, Bestrahlungsstärke und Modus die Werte an, welche vom Analog-Digital-Wandler ausgelesen und intern wiederholend berechnet werden. Die Taster dienen dazu, die Bestrahlungsstärke zu erhöhen, zu reduzieren und den Modus umzuschalten. Mit der nachfolgenden Beschreibung der Software wird das Prinzip des Entprellens, des automatischen Zählens in Zehnerschritten und die Funktionsweise der Modustaste beschrieben.

\subsubsection{Tasterentprellung}
%ISR (TIMER0_OVF_vect)
Der Timer-Interrupt im Mikrocontroller ruft alle 16ms eine Funktion auf, die den Taster entprellt, aber auch die Dauer des Tastendrucks registriert.
%button |= ~old_button & current_button & BUTTONMASK;
Der Taster (\textit{button}) wird als gedrückt erkannt, wenn der alte invertierte Zustand (\textit{old\_button}), der momentane Zustand (\textit{current\_button}) und die Pinmaske (\textit{BUTTONMASK}) übereinstimmen.
%if (button)
Mit logischen Bedingungen werden das Auf- und Abwärtszählen, das automatische Zählen und der Modus gewählt. Für das Inkrementieren und Dekrementieren von \textit{prozent} wird \textit{button} mit dem Tasterwert \textit{BUTTONUP} oder \textit{BUTTONDOWN} verglichen. Bei einem richtigen Vergleich setzt das Programm die Verzögerung des automatischen Zählers (\textit{autorepeat}) präventiv auf 50, was ungefähr zwei Sekunden entspricht und \textit{prozent} wird de- oder inkrementiert. Eine Prüfung von \textit{prozent} verhindert, das \textit{prozent} 20 unterschreitet oder 100 überschreitet.

\subsubsection{Automatisches Zählen in Zehnerschritten}
Nach den Bedingungen für das einfache Zählen folgen die Bedingungen für das automatische Zählen. Die Bedingung für das automatische Aufwärtszählen ist der momentan gedrückte Taster \textit{current\_button} UND-Verknüpft mit dem Tasterwert \textit{BUTTONUP}. Ergibt die Verknüpfung eine Eins ist die Bedingung erfüllt. Folglich wird der automatische Zählerwert ab dem gesetzten Verzögerungswert (\textit{AUTOREPEAT\_SET}) inkrementiert, solange der Taster gedrückt wird. Andernfalls wird unterbrochen und es wird nicht weitergezählt. Falls der Verzögerungswert Null wird, wird \textit{autorepeat} nochmals auf den Verzögerungswert gesetzt für eine allfällige Fortsetzung des automatischen Zählens. Bevor die eigentliche Berechnung erfolgt wird der Taster (\textit{button}) gleich dem Tasterwert (\textit{BUTTONUP} oder \textit{BUTTONDOWN}) gesetzt. Jetzt wird die nächstgrössere Zehnerzahl berechnet. Dabei wird die Eigenschaft von einem Integer verwendet, wonach bei beliebigen Operationen immer eine Ganzzahl herauskommt. So wird die Bestrahlungsstärke mit zehn dividiert und mit zehn multipliziert, dabei verliert die Bestrahlungsstärke die Ziffer an Einerposition. Anschliessend wird zehn addiert und die nächst grössere Zehnerzahl ist berechnet. Auch hier wird am Schluss geprüft, dass beim Aufwärtszählen die Grenze von 100 nicht überschritten wird.

Das automatische Abwärtszählen ist gleich realisiert bis zum Punkt der Berechnung. Die Berechnung braucht eine Hilfsvariable. Die Differenz zwischen der momentanen Bestrahlungsstärke und der errechneten Zehnerzahl nach der obigen Rechnung wird in die neue Variable gespeichert. Ist die Differenz zehn wird die obige Rechnungsmethode verwendet. Ist jedoch die Differenz grösser als zehn wird bei der Rechnungsmethode auf das Verringern um zehn verzichtet. Das bewirkt, dass z.B. die Zahl 55 beim automatisch verringern nicht 40 ergibt sondern 50. Zuletzt wird wieder geprüft, damit die Bestrahlungsstärke 20 nicht unterschreitet.

\subsubsection{Modus umschalten}
Entspricht der gedrückte Taster (\textit{button}) dem Tasterwert \textit{MODE} wird die Funktion (\textit{modusAendern}) aufgerufen, die in einer If-Else-Bedingung vom Modus \textit{SAUBER} in den Modus \textit{VERSCHMUTZT} umschaltet und umgekehrt. Beim Umschalten des Modus wurde auf ein automatisches Zählen verzichtet, weil keine grosse Auswahl getroffen werden muss mit dem Taster.

\subsubsection{Bildschirmsoftware}
Der Bildschirm wird beim Starten der Software initialisiert und durch die Funktion \textit{displayAktualisieren} mit den Startwerten beschrieben. Ist der erste Durchlauf abgeschlossen wird der Bildschirm alle 300ms aktualisiert.

Die erste Zeile ist für die Spannung in Volt reserviert. Sie wird im Format \#\#.\#\# angezeigt. Um die Spannung vom Typ Integer darzustellen wird in der Funktion \textit{printf} die Variable\textit{spannung} zweimal ausgegeben. Der Wert vor dem Komma wird durch 100 dividiert und der Wert nach dem Komma mit Modulo 100 berechnet. Zudem wird in der Darstellung der Spannung unterschieden, wenn sie grösser als 999 oder kleiner als 1000 ist. Diese Bedingung verhindert, dass der, auf dem Bildschirm, ausgegebene Text horizontal springt. \todo{Wird es wirklich 2x ausgegeben?}

Die zweite Zeile ist für den Strom in Ampere reserviert. Er wird im Format \#.\#\# angezeigt. Um den Strom vom Typ Integer darzustellen wird die gleiche Berechnung angewendet, wie bei der Spannung. Weil der Strom nicht über 999 hinausgeht braucht keine Zusatzbedingung wie bei der Spannung.

Die dritte Zeile ist für die Bestrahlungsstärke in Prozent reserviert. Sie wird im Format \#\#\# angezeigt. Der Wert \textit{prozent} wird  Eine Zusatzbedingung ändert \todo{Noch nicht fertig}